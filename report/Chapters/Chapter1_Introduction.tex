
\chapter{Introduction}

\label{Chapter1}

%----------------------------------------------------------------------------------------
%	SECTION 1
%----------------------------------------------------------------------------------------

\section{Motivation}
Human kind exerts an ever increasing pressure on natural and ecological systems due to the associated consequences of the explosion of human population. The exploitation of the earth manifests itself in extraction of natural ressources, proliferation of human-made infrastructure and waste, and increasing production land use for crop and pastry land \parencite{kareiva2007}. As a logical consequence, we observe widespread declines in biodiversity \parencite{newbold2015}, 
decrease in natural habitat, attrition of wilderness areas, deforestation, and enhanced emission of greenhouse gases to the atmosphere. This increasing intrusion leads to reduction of benefits that humans receive from natural systems \parencite{costanza2014} such as the extinction of natural ressources, and ultimately provoke natural disaster induced by effects such as climate change. 

An essential prerequisite to mitigate human threat to nature is the access to data that allows for spatial and temporal mapping of human activity \parencite{raiter2014}. To this end, the last decades have brought about developments of affordable and recurrent remote sensing technology \parencite{hansen2013}. In particular, we now have public and continuous access to overhead imagery data for earth observation in different levels of detail, ranging from 100m to 0.01m. Overhead imagery data is obtained either by satellites or by airborne sensor systems. Additionally, remote sensing technologies open up the road for applications in agricultue, disaster recovery, urban development, and environmental mapping.

The task of detecting various types of man-made structure and man-induced change has become a key problem in remote sensing image analysis. However, unlike the computer vision community that disposes of datasets with thousands or millions of images containing up to thousands of distinct annotations \parencite{everingham2010, deng2009, lin2014, krasin2016}, the remote sensing community is only recently making first steps towards creating standardized labelled large scale datasets. Several approaches into this direction have been focussing either on classifying land cover and land use \parencite{sumbul2019} or annotating overhead images with object categories \parencite{vanetten2018, lam2018}. These annotations were used to perform object detection or segmentation \parencite{yang2010, krasin2016}, and to e.g. map roads and buildings \parencite{vanetten2018, vanetten2019}. However, the statistics of the images and categories in these works is heavily biased towards man-made structures so that wilderness areas are strongly underrepresented. 

The computer vision community has largely benefited from recent advances in deep learning ultimately leading to the outsourcing of convolutional neural networks pretrained on massive datasets. In the remote sensing community, researchers are recently also starting to follow this pathway \parencite{sumbul2019}. However, pretrained models are not yet widely available, so that many works in the remote sensing field are based on fine-tuning neural networks  pre-trained on traditional computer vision tasks.

\section{Satellogic}
This work has been developed in cooperation with Satellogic, a company that provides earth observation data and analytics as a service to enable better decision making for industries, governments, and individuals. Satellogic was founded in 2010 in Buenos Aires, and has expanded since then with offices in Barcelona and China. Satellogic builds, launches and maintains their own satellites.

Satellogic focusses on developing heavily weight and cost optimized satellites. Their first nano satellite, called Capitán Beto, was sent to space in 2013 \parencite{wiki_satellogic}. Currently, Satellogic has 8 satellites orbiting space, whose weight is 35kg, a minuscul fraction of conventional satellite systems (more than 1000kg). The satellites feature hyperspectral image aquisition at 1m pixel resolution. Satellogic envisions to have 300 satellites orbiting the earth within a few years providing real time imagery for any geospatial location. 

The hyperspectral technology i.e. image aquisition capability in more than 30 spectral bands allows for monitoring the earth with great detail \parencite{satellogic2019}. Every object and every plant has its own spectral fingerprint. Measuring the optical reflectance to the solar radiation for instance allows to distinguish between different kinds of crops, and its status of irrigation and fertilization. Further, it is possible to measure the level of pollution in the air and monitor vegetation below the water surface. Satellogic's clients apply this technology to map land use, monitor infrastructure, track agricultural development, evaluate the health of crops, and evalute productivity of natural ressources.

At 1m pixel resolution, 10 satellites can remap 1 million square kilometers every 6 weeks. Note that the surface of the earth is roughly 500 million square kilometers of which about 30\% is land and 70\% is water, which brings us to the goals of this thesis.


\section{Thesis Goals and Outline}
The motivation for this Master's thesis is to provide an answer to the question: What is the optimal resolution to detect human impact in satellite imagery, having in mind the economical and ecological cost of acquiring and processing the information? The goal here is not to build the top performance, state-of-the-art model to detect all sorts of human impact in satellite images, but rather to analyze the feasibility and cost of doing so at different resolutions. Of course, better algorithms could be trained on larger datasets to accurately identify certain types of human impact, but we consider a more general problem.

To address this problem, we divide the work into three parts. First, we develop a detector that is capable of identifying man-made structures on two aerial imagery datasets that we collect and annotate. Next, we study the performance of the detector in terms of classification accuracy as a function of image resolution i.e. the resolution per pixel. In the last part, we provide an approximate estimation of the costs of the entire pipeline. These also include metrics related to building, launching and maintaining a satellite. The estimation is performed for the entire spectrum of resolutions.

The detailed outline of the thesis is the following:
\begin{itemize}
	\item Chapter~\ref{Chapter2} provides an overview of existing datasets and a detailed description of the construction of our own datasets. We further discuss the data manipulation pipeline.
	
	\item Chapter~\ref{Chapter3} gives an introduction to deep learning. We discuss theoretical concepts and recent advances in the field.
	
	\item Chapter~\ref{Chapter4} discusses the approach we followed to develop a detector capable of classifying man-made structures in aerial images.
	
	\item Chapter~\ref{Chapter5} presents the final results regarding the performance of the deep learning detector as a function of resolution for multiple image categories. It also provides an estimation of the cost to monitor the entire surface of the earth.
	
	\item Chapter~\ref{Chapter6} concludes the thesis and outlines potential next steps.	
\end{itemize}	