
\chapter{Conclusions}

\label{Chapter6}

%----------------------------------------------------------------------------------------

In this final chapter we discuss and close the different aspects of the project, from the problem definition itself, to the datasets build and models developed, and we conclude our thesis with some further work ideas to continue and enhance our approach.

\section{The Problem}

The intial phase during the development of the project consisted on clearly defining the problem to study. The goal was to investigate which satellite imagery resolutions allowed for an accurate detection of man-made structures, and what would be the cost associated. For that, we needed to define the scope of human activity to consider, look for suitable datasets for this study (which eventually lead to building our own datasets), and define the actual technical problem to be modeled, in order to evaluate the accuracy by resolution. 

Both for the datasets and the problem, we needed them to be feasible enough to not require high technical and computational efforts (which would be, for instance, trying to detect every type of human activity in the images, providing their position and shape, and classifying them into several more categories). On the other hand, we needed it to be a realistic situation, so the results obtained could be extrapolated to other, more complex scenarios.

All in all, having a well-defined problem scope and a good approach to tackle it allowed us to achieve remarkable and realistic results.

\section{The Datasets}

After investigating existing datasets of labeled satellite images, we could not find the one suitable for our purpose, as most of them were mainly focused towards urban areas, or were just build for some other different goal that would not work for us. Hence, we decided to build it ourselves. It had to be representative enough to pose a challenge for our models, yet feasible to be build with our available time and resources. The four categories considered (agriculture, shrubland-grassland, semi-desert, and forest-woodland) and the balance between non-existent and existent human impact images allowed us to build a good and representative dataset, which makes reasonable to eventually generalize the results obtained to other scenarios.

Of course, we acknowledge that having a larger dataset of images, annotated with higher degree of detail, like position, shapes and types of man-made or natural structures, would be great to build high performance models, capable of detecting all sorts of human impact with far better accuracy. Nevertheless, this high-precision goal could not be fitted into our general purpose.

\section{The Models and Results}

Using a pre-trained CNN like the ResNet helped us to speed up the training process and achieve good results without requiring a very large dataset and computational power. The binary classification problem considered turned out to be feasible and representative of how accuracy is affected with a decrease in the image resolution. 

From the results we observe that, as expected, the higher the resolution the better, but also that there seems to be sweet spot between $1$-$2m$ and $8m$ where, except for the more subtle forms of human activity, most of the man-made structures studied are detectable with good accuracy. This trade-off with the resolution allows to consider more cost-economic satellite solutions without dramatically compromising accuracy and utility. For instance, operating a satellite at $2m$ (or $8m$) instead of $1m$ reduces the cost approximately by a factor 6 (or $100$).

\section{Further work}

With all that said, we realize that there is still plenty of space for further work and investigations, so let us now indicate some of these ideas.

First of all, having a better dataset could help improving the investigations and opening new lines to explore. It could be improved and enlarged with more variate images, with a better and more consistent classification, and including more detailed annotations of the position and type of objects or structures appearing. This would allow to train more accurate models capable of detecting all these kind of human impact.

Regarding the model, other techniques for feature extraction could be studied, like other pre-trained Neural Networks, and the parameters itself (like the number of activations to consider, the architecture of the model or the training phase) could be further fine-tuned. Additionally, the pipeline could be made more robust so that it could ingest a larger amount of data, as part of the improvements suggested for the dataset. And, of course, having a powerful computational cluster would allow to speed up the processes and target more ambitious goals.

A more in depth study of the results could help to understand more precisely on which images the algorithms fail, what kind of information are learning (patterns, colors, shapes, etc) and how to enhance them.

Finally, it would be interesting to have a more detailed analysis of the costs associated to all this solution, from data gathering, processing and modeling to the production implementation itself. Also, taking into account other related factors, like infrastructure needed, legal aspects and ecological footprint \parencite{Strubell2019} would give a more complete idea of the viability of global satellite image analysis.

\

In conclusion, this project has allowed us to investigate a relatively new topic, covering from the data gathering to the technical implementation using state-of-the-art tools, and leaving the door open to further investigations.
