
\chapter{Conclusions}

\label{Chapter6}

%----------------------------------------------------------------------------------------

\section{The Problem}

The intial phase of the project consisted on clearly defining the problem to study. The goal was to investigate which satellite imagery resolutions allowed for an accurate detection of man-made structures (or human activity), and which would be the cost associated. Hence, we needed to define the scope of human impact to consider, look for datasets suitable for this need (and eventually build our own datasets), and define the technical problem to be modeled, in order to evaluate the accuracy at each resolution. Both for the datasets and the problem, we needed them to be feasible enough to not require high technical and computational efforts, which for instance would be the case of detecting every single human activity in the images, providing their position and shape, and classifying into several categories. On the other hand, it should be a realistic enough situation, so the results obtained could be extrapolated to other, more complex scenarios.

\section{The Datasets}

As we could not find a dataset suitable for our purpose, we decided to build our own dataset. It had to be representative enough to be a challenge for our models, yet feasible to be build. Of course, having annotated images with high degree of detail, like position, shapes and types of man-made or natural structures, would be great to build high performance models.

All in all, ...

\section{The Models}

- Classification problem considered

- Feasability

- Great accuracy, decreasing for lower resolutions

\section{Further work}

\begin{itemize}
	\item Dataset: improve it, enlarge with more and more variate images, better classify them, even anotate with more categories of actual objects appearing, consider annotating areas, and train a model to detect this particular patterns. Yet, all this would diverge from the original scope of the project,
	\item Model: CNN feature extraction, other architectures and number of activations, layers, neurons, training epochs. Make a more robust pipeline, load images on batches, so a larger dataset can be considered. Implement some sort of data augmentation if needed
	\item Results: further study on which images the models fail, what kind of information is learning (patters, colors, shapes, etc)
	\item Conclusions: further analysis costs of such implementations, environment impact, etc
\end{itemize}