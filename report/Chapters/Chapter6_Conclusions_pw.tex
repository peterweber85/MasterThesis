
\chapter{Conclusions}

\label{Chapter6}

%----------------------------------------------------------------------------------------

In this final chapter we discuss and close the different aspects of the project, from the problem definition itself, to the build datasets and developed models. We conclude our thesis with proposals and ideas on how to continue and enhance our approach.

\section{The Problem}

The intial phase during the development of the project consisted of clearly defining the problem to study. The goal was to investigate which satellite imagery resolutions allowed for an accurate detection of man-made structures, and what would be the associated cost. For that, we needed to define the scope of human activity to consider, look for suitable datasets for this study (which eventually lead to building our own datasets), and define the actual technical problem to be modeled, in order to evaluate the accuracy by resolution. Regarding the latter point we made our first big step when we decided to change the resolution by downsampling the images i.e. reducing the number of pixels but keeping the physical area of the image constant. This was not obvious a priori.

Both for the datasets and the problem, we needed them to be feasible enough to not require high technical and computational efforts, which would be, for instance, trying to detect every type of human activity in the images, providing their position and shape, and classifying them into several categories. On the other hand, we needed it to be a realistic situation, so the obtained results could be extrapolated to other, more complex scenarios.

Overall, having a well-defined problem scope and a good approach to tackle it allowed us to achieve reasonable and realistic results.

\section{The Datasets}

The publicly available datasets we were able to find were not suitable for our study, because they were heavily biased towards urban areas and non of them was constructed with a similar purpose than ours. We therefore decided to build our own datasets. It had to be representative enough to pose a challenge for our models, yet feasible to be build with our available time and resources. The four considered categories (agriculture, shrubland-grassland, semi-desert, and forest-woodland) and the balance between non-existent and existent human impact images allowed us to build a good representative dataset, which makes it reasonable to generalize the results to larger and more complex datasets.

Of course, we acknowledge that having a larger dataset of images, annotated with higher degree of detail, like position, shapes and types of man-made or natural structures, would be great to build high performance models, capable of detecting human impact with even better accuracy. Nevertheless, this high-precision goal could not be fitted into our general purpose.

\section{The Models and Results}

Using a pre-trained CNN like the ResNet helped us to speed up the training process and achieve good results without requiring a very large dataset and computational power. The binary classification problem that we considered turned out to be feasible and representative of how accuracy is affected with a decrease in the image resolution. 

From the results we observe that, as expected, the higher the resolution, the better, but also that there seems to be a sweet spot between $1$-$2m$ and $8m$ where, except for the more subtle forms of human activity, most of the studied man-made structures are detectable with good accuracy. This trade-off with the resolution allows to consider more cost-economic satellite solutions without dramatically compromising accuracy and utility. For instance, operating a satellite at 2m (8m) instead of 1m reduces the cost approximately by a factor 6 (100).

\section{Further work}

With all that said, we realize that there is still plenty of space for further work and investigations, so let us now indicate some of these ideas.

First of all, having a better dataset could help improving the investigations and opening new lines to explore. It could be improved and enlarged with more variate images, with a better and more consistent classification, and including more detailed annotations of the position and type of objects or structures appearing. This would allow to train more accurate models capable of detecting all these kind of human impact.

Regarding the model, other techniques for feature extraction could be studied, like other pre-trained Neural Networks. Also the parameters itself could be further fine-tuned, like the number of activations to consider, the architecture of the model and the training phase. Additionally, the pipeline could be trained on a proper cloud infrastructure so that the amount of data we are able to use would not be limited by Google Colab's memory restrictions. This would allow for a much more robust pipeline when ingesting a larger amount of data.

A more in depth study of the results could help to understand more precisely on which images the algorithms fail, what kind of information are learning (patterns, colors, shapes, etc) and how to enhance them.

Finally, it would be interesting to have a more detailed analysis of the associated costs to all this solution, from data gathering, processing and modeling to the implementation itself. Also, taking into account other related factors, like needed infrastructure, legal aspects and ecological footprint \parencite{Strubell2019} would give a more complete idea of the viability of global satellite image analysis.
