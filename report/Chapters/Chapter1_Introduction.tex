
\chapter{Introduction}

\label{Chapter1}

%----------------------------------------------------------------------------------------
%	SECTION 1
%----------------------------------------------------------------------------------------

\section{Motivation}
Human kind exerts an ever increasing pressure on natural and ecological systems due to the associated consequences of the explosion of human population. The exploitation of the earth manifests itself in extraction of natural ressources, proliferation of human-made infrastructure and waste, and increasing production land use for crop and pastry land. As a logical consequence, we observe widespread declines in biodiversity, 
decrease in natural habitat, attrition of wilderness areas, deforestation, and enhanced emission of exhaust (CO2) to the atmosphere. This increasing intrusion leads to reduction of benefits that humans receive from natural systems, such as extinction of natural ressources, and ultimately provoke natural disaster induced by effects such as climate change. \textcolor{red}{CITE}

The last decades have brought about developments of affordable and recurrent remote sensing technology. In particular, we now have public and continuous access to overhead imagery data for earth observation in different levels of detail, ranging from 100m to 0.01m. Overhead imagery data is obtained either by satellites or by airborne sensor systems. This data allows for spatial and temporal mapping of human activity, which is an essential prerequisite to mitigate human threat to nature. Further, remote sensing technologies open up the road for applications in agricultue, disaster recovery, urban development, and environmental mapping.

Satellites paragraph Satellites paragraph Satellites paragraph Satellites paragraph Satellites paragraph Satellites paragraph Satellites paragraph Satellites paragraph Satellites paragraph Satellites paragraph Satellites paragraph Satellites paragraph Satellites paragraph Satellites paragraph Satellites paragraph Satellites paragraph Satellites paragraph Satellites paragraph Satellites paragraph Satellites paragraph Satellites paragraph Satellites paragraph Satellites paragraph Satellites paragraph Satellites paragraph Satellites paragraph Satellites paragraph Satellites paragraph.

The task of detecting various types of man-made structure and man-induced change has become a key problem in remote sensing image analysis. However, unlike the computer vision community, that disposes of datasets with millions of images and thousands of distinct annotations, the remote sensing community is struggling to create standardized labelled large scale datasets. There have been several approaches into this direction either classifying land cover and land use or object detection on overhead images.


\begin{itemize}
\item Possibilities of using overhead imagery, quickly mention approaches and datasets, object detectio scene detection  land use detection Advances in computer vision that make all of this possible
\item Short introduction to satellites Satellogic and capabilities of collecting and processing imagery data. Introduce cost and environmental impact
\item thesis outline
\end{itemize}



BigEarthNet Motivation:
Existing Remote Sensing Datasets contain a small number of annotated images and therefore do not suffice to train a complex ConvNet with many different parameters.
Remote sensing versus Computer (Using pretrained nets on ImageNet)


BagOfVisualWords: useless

Extreme value theory-based calibration: useless

Human footprint: Climate Change, population control, ilegal habitat/ilegal land use

xView: The abundance of overhead image data from
satellites and the growing diversity and significance of real-world applications enabled by that
imagery provide impetus for creating more sophisticated and robust models and algorithms for object detection.


The goal is not really to build the top performance, state-of-the-art model to detect all sort of human impact in satellite images, but rather to analyze the feasibility and cost of doing so at different resolutions. Of course, better algorithms could be implemented to accurately identify certain types of human impact, but we consider a more general problem.

Thesis proposal of Jordi: With the development of affordable and recurrent remote sensing technology, we can now access frequent geospatial information in different levels of detail, ranging from 100m to 0.01m. The task of detecting various types of man-made structure and man-induced change has become a key problem in remote sensing image analysis.
This project is focused on providing an answer to a simple question: what is the minimum resolution required to detect the different man-made structures (roads, buildings, cars, pipes, crop fields, etc.) in remote sensing images? Determining this value is important not only for designing optimal satellite sensors (in terms of cost vs image information) but also to use optimal data sources (also in terms of cost vs image information) when developing data-based remote sensing products. At a global level, this knowledge contributes to understand the impact of our species on the planet.
The approach will be based on the analysis of the statistical properties of remote sensing images. There are two expected outcomes of the analysis: (i) the definition of an image signature to discriminate between natural and different man-made structures and (ii) experiments showing which is the optimal resolution.
The study of the statistical properties of natural images belonging to different categories and their relevance for scene and object categorization tasks has been an active field of research for several decades. Results show how visual categorization based directly on low-level features, without grouping or segmentation stages, can benefit object localization and identification. In this project, we want to extend this study to remote sensing images belonging to man-made and natural structures. Taking for granted that detection methods will be based in deep learning, we will use deep learning low level detectors (extracted from convolutional neural networks) to define image signatures.




\section{Satellogic}


\section{Thesis Outline}